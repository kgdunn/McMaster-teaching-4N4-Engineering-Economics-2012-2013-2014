%!TEX root = ../4N4-course.tex

\begin{frame}\frametitle{Plan for today's class}
	\begin{enumerate}
		\item	Background
		\item	Administrative issues
		\item	Course content
		\item	Short brainstorming session of topics to cover
	\end{enumerate}
\end{frame}

\begin{frame}\frametitle{Credits}	
	\begin{itemize}
		\item	Dr. Don Woods, Dr. Thomas Marlin, Dr. Prashant Mhaskar
		\item	Taught the course since the 1980's
		\item	Course outline and topics covered are similar to theirs
		\item	We will use their notes and materials for most of the course
	\end{itemize}	
\end{frame}

\begin{frame}\frametitle{Background}
	{\color{myGreen}{About myself}}
	\begin{itemize}
		\item	Undergraduate degree from University of Cape Town, 1999
		\item	Masters degree from McMaster, 2002 (not a ``doctor'', please)
		\item	Worked with a number of companies since then on data analysis and consulting projects
		\item	Worked at GSK on a 1-year contract until June 2012		
		\item	Now working full-time at McMaster since July 2012
		\item	Drop-in hours: Tuesday and Thursday afternoons
		\item	Office is in BSB, room B105
		\item	Arrange a meeting: \url{kevin.dunn@mcmaster.ca}
		\item	Cell: (905) 921 5803
		\item	extension 27337
	\end{itemize}	
\end{frame}

\begin{frame}\frametitle{Teaching assistants}
	{\color{myGreen}{Alicia Pascall }}
	\begin{itemize}
		\item	\url{pascalaa@mcmaster.ca}
		\item	JHE, room 370 (upstairs)
		\item	extension 22008
		\item	Currently doing her Masters with Tom Adams
	\end{itemize}
	{\color{myGreen}{Yasser Ghobara}}
	\begin{itemize}
		\item	\url{ghobary@mcmaster.ca}
		\item	JHE, room 369
		\item	extension 24031
		\item	Currently doing his Masters with Chris Swartz
	\end{itemize}
	\vspace{12pt}
	Office hours will be arranged.
\end{frame}

\begin{frame}\frametitle{Video and audio recordings}
	\begin{itemize}
		\item	As long as feasible, I will try to video record all classes
		\item	Useful if you miss a class
		\item	Quality might not be the best
		\item	Usually available 24 to 48 hours after the class
		\item	Audio recordings will also be made available, when possible
	\end{itemize}
\end{frame}

\begin{frame}\frametitle{Course website}	
	\begin{exampleblock}{}
		\centering 
		\href{http://learnche.mcmaster.ca/4N4}{http://learnche.mcmaster.ca/4N4}
	\end{exampleblock}
	\begin{itemize}
		\item	Please check \textbf{every day} for announcements (top left)
		\item	Slides might added to the site before class
		\item	Please print slides and bring to class
		\item	Notes will be ready on the Monday before Tuesday's class
		\item	Tutorials and assignments will be posted by Friday for the Monday tutorial
	\end{itemize}
\end{frame}

\begin{frame}\frametitle{References and readings}
	\vspace{12pt}

	There are too many references to mention here. Please see the printed notes and the course website.
	
	\vspace{12pt}
	
	You will have to be selective in terms of what you download, buy and spend your time reading.
\end{frame}

\begin{frame}\frametitle{Course feedback via Learning website}
	\begin{itemize}
		\item	I might not have explained something clearly;  
		\item	you didn't get a chance to ask a question, \emph{etc}		
	\end{itemize}
	\href{http://learnche.mcmaster.ca/feedback-questions}{http://learnche.mcmaster.ca/feedback-questions}
	\vspace{12pt}
	\hrule
	\begin{center}
		\includegraphics[width=0.65\textwidth]{\imagedir/teaching/anonymous-feedback.png}
	\end{center}
	\hrule
\end{frame}

\begin{frame}\frametitle{Expectations outside class}
	\begin{itemize}
		\item	You can expect TAs and I to answer emails promptly
		\item	If you have questions
			\begin{enumerate}
				\item	Please email the TA with CC to me \hfill {\tiny{\color{myOrange}{$\longleftarrow$ hopefully this solves your problem}}}
				\item	if not, set up meeting with TA or myself
			\end{enumerate}
		\item	Please email from your McMaster address (filtering)
	\end{itemize}
\end{frame}

\begin{frame}\frametitle{What we will cover}
	\todo{Mindmap here}
	* Engineering economics:
		* Declining value of money over time
		* Profitability: capital cost, working capital and manufacturing costs
		* Making fair cost comparisons
		* Depreciation
		* Capital costing
		* Sensitivity to various scenarios
	* Safety: e.g. tools such as pressure relief; HAZOP	
	* Troubleshooting
	* SDL project
	* Process Operability
\end{frame}

\begin{frame}\frametitle{Second half of course}
	Project application focused
	Graded workshops: pass or fail
\end{frame}

\begin{frame}\frametitle{Grading}
	\todo{Messy formula in the course outline}
	
	\begin{itemize}
		\item	\emph{Grading allocation is subject to change}
		\item	Course letters will be assigned using standard system
	\end{itemize}
\end{frame}

\begin{frame}\frametitle{Midterms and exam}
	
	There is a midterm on the Process Economics section.
	\vspace{12pt}
	There will be a final exam.
	\vspace{24pt}
	
	All tests and exams:
	\begin{itemize}
		\item	open notes -- any form of paper
		\item	any calculator
		\item	no e-books (any ideas on how to handle this? feedback form)
	\end{itemize}	
\end{frame}

\begin{frame}\frametitle{Project}
	A major component of the course is the self-directed learning (SDL) project.
\end{frame}

\begin{frame}\frametitle{Self-directed learning}
	Any examples of self-directed learning you've succeeded at recently?
	\begin{itemize}
		\item	learned how to ride a motorcycle
		\item	\iftoggle{instructor}{learn how to fix drywall}{}
		\item	\iftoggle{instructor}{change your car's oil}{}		
		\item	\iftoggle{instructor}{unclog drains in your house}{}
		\item	\iftoggle{instructor}{cook an ethnic food dish to impress your date}{}
		\item	\iftoggle{instructor}{something learned during your co-op work term?}{}
		\item	\iftoggle{instructor}{process of buying a car or a house}{}
		\item	\iftoggle{instructor}{plant, grow and maintain your own vegetables}{}
		\item	\iftoggle{instructor}{learn a new language for travel purpose}{}
		\item	\iftoggle{instructor}{start your own company and run it: what is required?}{}
		\item	\iftoggle{instructor}{am I better off buying a new car or a used car?}{}
	\end{itemize}
	\vspace{12pt}
	Our first tutorial on Monday: personal finance
\end{frame}

\begin{frame}\frametitle{Why self-directed learning is important}
	\begin{itemize}
		\item	No employer has an obligation to employ you until retirement
		\item	You must take action to make yourself employable:
		\begin{itemize}
			\item	co-op terms
			\item	job market is looking for new hires and you have the profile they are looking for
			\item	you start your own company
		\end{itemize}
		\pause
		\item	You will likely be a contract-based employee for the next few years (I'm still a contracted employee)
		\item	Your degree is no guarantee for employment: but it helps tremendously: {\color{myRed}{{\small you don't need to remain in chemical engineering!}}}
		\pause
		\item	Once you start working: \textbf{keep yourself employable}
		\begin{itemize}
			\item	technologies are used that were never taught in engineering
			\item	new technologies will be invented
			\item	different ways of doing business			
		\end{itemize}
	\end{itemize}
\end{frame}

\begin{frame}\frametitle{SDL in this course}
	Self-directed learning is a process: 
	\begin{itemize}
		\item	identify what you already know
		\item	recognize where you don't have the knowledge
		\item	diagnose what you need
		\item	figure out how to meet your needs (which resources to use?)
		\item	implement it: go to the library/internet/meet with people
		\item	evaluate the success: did it work? do I need to start over again?
	\end{itemize}
	\vspace{12pt}
	You'll be doing this in groups on topics related to \texttt{4N4}: two-for-one
\end{frame}

\begin{frame}\frametitle{Other issues}
	Cover letter
	Grammar, spelling and other details
\end{frame}

\begin{frame}\frametitle{Other skills you will learn}
	\begin{itemize}
		\item	Presentation skills: e.g. in tutorials you will present your solution to the class
		\item	Being a group chairperson
		\item	Dealing with (dys)functional groups
		\item	Being an effective group member
		\item	Troubleshooting and getting to the root cause (use of case studies)
		\item	Improve your technical writing skills
		\item	Learning on your own
		\item	Reading and interpreting economic data
		\item	Introduction to engineering ethics
		\item	Time management and project management
	\end{itemize}
	
\end{frame}

\begin{frame}\frametitle{Why group-based work}
	See: Turton, Ch 28
\end{frame}

\begin{frame}\frametitle{Group selection}
	How the process works
	Deadlines
\end{frame}

\begin{frame}\frametitle{Some unsolicited advice}
	Recognize that the lectures are about 25\% of the work for this course

	Final year is a tough year: many competing demands on your time, especially in 2nd semester

	Some useful tips:
	* exercise a few times per week: walk around campus, Pulse, group class, cardio circuit, jog, push-ups regularly
	* prioritize events in your social life and academic life
	* speak with your group members all the time about how feel: communicate communicate communicate
	* sleep
	* eat well: veggies and balance meals
	* overall time-management
\end{frame}

\begin{frame}\frametitle{Over to you ...}
	Please tell me more about your hopes and expectations on the handout.
\end{frame}