%!TEX root = ../4N4-course.tex

\begin{frame}\frametitle{Plan for today's class}
	\begin{enumerate}
		\item	Background
		\item	Administrative issues
		\item	Review of the course
	\end{enumerate}
\end{frame}

\begin{frame}\frametitle{Credits}	
	\begin{exampleblock}{}
		\centering {\color{myOrange}{Dr. Don Woods}} and {\color{myBlue}{Dr. Thomas Marlin}}
	\end{exampleblock}
	\begin{itemize}		
		\item	What do they have in common?
		\item	Have been the main instructors since the 1980's
		\item	Course outline and topics covered are similar to theirs
		\item	We will use their notes, slides, and other materials for most of the course
	\end{itemize}	
\end{frame}

\begin{frame}\frametitle{Background}
	{\color{myGreen}{About myself}}
	\begin{itemize}
		\item	Undergraduate degree from University of Cape Town, 1999
		\item	Masters degree from McMaster, 2002 (not a ``doctor'', please)
		\item	Worked with a number of companies since then on data analysis and consulting projects
		\item	Worked at GSK on a 1-year contract until June 2012		
		\item	Now working full-time at McMaster since July 2012
		\item	Drop-in hours: Tuesday and Thursday afternoons
		\item	Office is in BSB, room B105
		\item	Arrange a meeting: \url{kevin.dunn@mcmaster.ca}
		\item	Cell: (905) 921 5803
		\item	extension 27337
	\end{itemize}	
\end{frame}

\begin{frame}\frametitle{Teaching assistants}
	{\color{myGreen}{Alicia Pascall }}
	\begin{itemize}
		\item	\url{pascalaa@mcmaster.ca}
		\item	JHE, room 370 (upstairs)
		\item	extension 22008
		\item	Currently doing her Masters with Tom Adams
	\end{itemize}
	\vspace{12pt}
	{\color{myGreen}{Yasser Ghobara}}
	\begin{itemize}
		\item	\url{ghobary@mcmaster.ca}
		\item	JHE, room 369
		\item	extension 24031
		\item	Currently doing his Masters with Chris Swartz
	\end{itemize}
	\vspace{12pt}
	Office hours will be arranged next week.
\end{frame}

\begin{frame}\frametitle{Video and audio recordings}
	\begin{itemize}
		\item	As long as feasible, I will try to video record all classes
		\item	Useful if you miss a class
		\item	Quality might not be the best
		\item	Usually available 24 to 48 hours after the class
		\item	Audio recordings will also be made available, when possible
	\end{itemize}
\end{frame}

\begin{frame}\frametitle{References and readings}
	
	There is a printed course pack.
	\vspace{12pt}
	\begin{itemize}
		\item	Buy it for \$37 from Print Factory Ink, (corner of Sterling and Main, Westdale), \emph{\textbf{or}}
		\item	Print out the PDFs from the course website
	\end{itemize}
	\vspace{12pt}
	There are too many textbooks and references to mention here. See printed notes and the course website for suggestions.
	
	\vspace{6pt}
	You will have to be selective in terms of what you download, buy and spend your time reading.
\end{frame}

\begin{frame}\frametitle{Course website}	
	\begin{exampleblock}{}
		\centering 
		\href{http://learnche.mcmaster.ca/4N4}{http://learnche.mcmaster.ca/4N4}
	\end{exampleblock}
	\begin{itemize}
		\item	Please check \textbf{every day} for announcements {\tiny (top left)}
		\item	Sometimes slides to supplement the course pack will be available. 
		\item	Check the day before class: won't be critical to bring, unless specifically mentioned.
		\item	Tutorials and assignments will be posted by Friday evening for the Monday tutorial slot.
	\end{itemize}
\end{frame}

\begin{frame}\frametitle{Course feedback via Learning website}
	\begin{itemize}
		\item	I might not have explained something clearly;  
		\item	you didn't get a chance to ask a question, \emph{etc}		
	\end{itemize}
	\href{http://learnche.mcmaster.ca/feedback-questions}{http://learnche.mcmaster.ca/feedback-questions}
	\vspace{12pt}
	\hrule
	\begin{center}
		\includegraphics[width=0.65\textwidth]{\imagedir/teaching/anonymous-feedback.png}
	\end{center}
	\hrule
\end{frame}

\begin{frame}\frametitle{Instructor, TA and student relationship}
	\begin{exampleblock}{}
		The relationship is one of {\color{myRed}{\emph{managers and colleagues}}}
	\end{exampleblock}
	
	\begin{itemize}
		\item	You can expect TAs and I to answer emails promptly
		\item	If you have questions
			\begin{enumerate}
				\item	Please email the TA with CC to me \hfill {\tiny{\color{myOrange}{$\longleftarrow$ hopefully this solves your problem}}}
				\item	if not, set up meeting with TA or myself
			\end{enumerate}
		\item	Please email from your McMaster address (filtering)
	\end{itemize}
\end{frame}

\begin{frame}\frametitle{\texttt{4N4} in context}
	\begin{center}
		\includegraphics[width=\textwidth]{\imagedir/econ-prob-solve/outcomes/4N4-scope-in-terms-of-ChemEng.png}
	\end{center}
\end{frame}

\begin{frame}\frametitle{More on the ``Design sequence''}
	\begin{center}
		\includegraphics[width=\textwidth]{\imagedir/econ-prob-solve/outcomes/4N4-course-context.png}
	\end{center}
\end{frame}

\begin{frame}\frametitle{What does this course cover {\scriptsize you'll learn to look at flowsheets in a different way}}
	\begin{center}
		\includegraphics[width=\textwidth]{\imagedir/econ-prob-solve/outcomes/look-at-a-flowsheet-differently.png}
	\end{center}
\end{frame}

\begin{frame}\frametitle{What we will cover}
	\begin{center}
		\includegraphics[width=\textwidth]{\imagedir/econ-prob-solve/outcomes/econ-prob-solve-4N4-mindmap.png}
	\end{center}
\end{frame}

\begin{frame}\frametitle{This course is preparation for engineering practice}
	\begin{itemize}
		\item	Professional Attitude 
		\begin{itemize}
			\item	I am responsible for my learning
			\item	I strive for many objectives (safety, reliability, \emph{etc}) in practice
		\end{itemize}
		\item	Professional Skills (more on this later)
		\item	Technical Knowledge
		\begin{itemize}
			\item	Build on engineering science with new knowledge and practical applications
		\end{itemize}
	\end{itemize}
\end{frame}

\begin{frame}\frametitle{Grading}
	\begin{center}
		\includegraphics[width=0.8\textwidth]{\imagedir/econ-prob-solve/outcomes/4N4-grading-2012.png}
	\end{center}
	\begin{itemize}
		\item	\emph{Grading allocation is subject to change}
		\begin{itemize}
			\item	Mid term exam: 10\%
			\item	Final exam: 25\%
		\end{itemize}
		\item	Course letters will be assigned using standard system
	\end{itemize}
\end{frame}

\begin{frame}\frametitle{Midterms and exam}
	
	\begin{itemize}
		\item	Midterm: process economics section
		\item	Final exam: on everything, including economics
	\end{itemize}
	\vspace{24pt}
	All tests and exams:
	\begin{itemize}
		\item	open notes -- any form of paper
		\item	any calculator
		\item	no e-books
	\end{itemize}	
\end{frame}

\begin{frame}\frametitle{Tutorials}
	Tutorial slots \textbf{A} (\emph{morning}) and \textbf{B} (\emph{afternoon}) on Mondays:
	
	\begin{itemize}
		\item	one question from the tutorial will be due as a group hand-in (no cover page)
		\item	group presentations of tutorial questions in the last half-hour (for grades)
		\item	the remaining questions will be due on Thursday (with cover page)
		\item	pop quizzes will likely be part of the tutorial slot
	\end{itemize}
\end{frame}

\begin{frame}\frametitle{During tomorrow's class}
	\begin{itemize}
		\item	Review the SDL project requirement
		\item	Talk about group work
		\item	You will complete a questionnaire about yourself, your goals and select group members
	\end{itemize}
\end{frame}


% \begin{frame}\frametitle{SDL Project}
% 	A major component of the course is the self-directed learning (SDL) project.
% \end{frame}
% 
% \begin{frame}\frametitle{Self-directed learning (SDL)}
% 	Any examples of self-directed learning you've succeeded at recently?
% 	\begin{itemize}
% 		\item	learned how to ride a motorcycle
% 		\item	\iftoggle{instructor}{learn how to fix drywall}{}
% 		\item	\iftoggle{instructor}{change your car's oil}{}		
% 		\item	\iftoggle{instructor}{unclog drains in your house}{}
% 		\item	\iftoggle{instructor}{cook an ethnic food dish to impress your date/partner}{}
% 		\item	\iftoggle{instructor}{something learned during your co-op work term?}{}
% 		\item	\iftoggle{instructor}{going through process of buying a car or a house}{}
% 		\item	\iftoggle{instructor}{plant, grow and maintain your own vegetables}{}
% 		\item	\iftoggle{instructor}{learn a new language for travel/pleasure}{}
% 		\item	\iftoggle{instructor}{start your own company and run it: what is required?}{}
% 		\item	\iftoggle{instructor}{figure out if I'm better off buying a new car or a used car?}{}
% 	\end{itemize}
% 	\vspace{12pt}
% 	Our 1st tutorial, next Monday, on personal finance: {\small \color{myOrange}{completely SDL}}
% \end{frame}
% 
% \begin{frame}\frametitle{Why self-directed learning is important}
% 	\begin{itemize}
% 		\item	No employer has an obligation to employ you until retirement
% 		\item	You must take action to make yourself employable:
% 		\begin{itemize}
% 			\item	co-op terms
% 			\item	job market is looking for new hires and you have the profile they are looking for
% 			\item	you start your own company
% 		\end{itemize}
% 		\pause
% 		\item	You will likely be a contract-based employee for the next few years (I'm still a contracted employee)
% 		\item	Your degree is no guarantee for employment: but it helps tremendously: {\color{myRed}{{\small you don't need to remain in chemical engineering!}}}
% 		\pause
% 		\item	Once you start working: \textbf{keep yourself employable}
% 		\begin{itemize}
% 			\item	technologies are used that were never taught in engineering
% 			\item	new technologies will be invented
% 			\item	different ways of doing business			
% 		\end{itemize}
% 	\end{itemize}
% \end{frame}
% 
% \begin{frame}\frametitle{SDL in this course}
% 	Self-directed learning is a process: 
% 	\begin{itemize}
% 		\item	identify what you already know
% 		\item	recognize where you don't have the knowledge
% 		\item	diagnose what you need
% 		\item	figure out how to meet your needs (which resources to use?)
% 		\item	implement it: go to the library/internet/meet with people
% 		\item	evaluate the success: did it work? do I need to start over again?
% 	\end{itemize}
% 	\vspace{12pt}
% 	You'll be doing this in groups on topics related to \texttt{4N4}: two-for-one
% \end{frame}
% 
% \begin{frame}\frametitle{SDL project}
% 	Apply what you have learned and are currently learning to an process.
% 	\vspace{12pt}
% 	\begin{itemize}
% 		\item	economics
% 		\item	operability issues: startup and shut down
% 		\item	safety
% 		\item	troubleshooting
% 		\item	summarize your learning (reflection on what was learned)
% 	\end{itemize}
% \end{frame}
% 
% \begin{frame}\frametitle{Other skills you will learn}
% 	\begin{itemize}
% 		\item	Setting goals
% 		\item	Presentation skills: e.g. in tutorials you will present your solution to the class
% 		\item	Being a group chairperson
% 		\item	Dealing with (dys)functional groups
% 		\item	Being an effective group member
% 		\item	Finding reliable learning materials
% 		\item	Troubleshooting and getting to the root cause (use of case studies)
% 		\item	Improve your technical writing skills
% 		\item	Learning on your own
% 		\item	Reading and interpreting economic data
% 		\item	Introduction to engineering ethics
% 		\item	Time management and project management
% 		\item	Entrepreneurship
% 		\item	Comfortable with engineering drawings
% 		\item	Dealing with ambiguity and uncertainty		
% 		\item	Communication: Cover letters, grammar, spelling
% 	\end{itemize}	
% \end{frame}
% 
% \begin{frame}\frametitle{Working in groups: less of this}
% 	\begin{center}
% 		\includegraphics[width=\textwidth]{\imagedir/econ-prob-solve/groups/working-in-groups-1.png}
% 	\end{center}
% \end{frame}
% 
% \begin{frame}\frametitle{Groups: we hope you only experience this}
% 	\begin{center}
% 		\includegraphics[width=0.85\textwidth]{\imagedir/econ-prob-solve/groups/working-in-groups-2.png}
% 	\end{center}
% \end{frame}
% 
% \begin{frame}\frametitle{Why group-based work}
% 	\begin{itemize}
% 		\item	$1+1+1+1+1 = 8$: magnify your strengths
% 	\end{itemize}
% 	See: Turton, Ch 28
% \end{frame}
% 
% \begin{frame}\frametitle{Group selection}
% 	\begin{itemize}
% 		\item	How the process works
% 		\item	Deadlines
% 	\end{itemize}
% \end{frame}
% 
% \begin{frame}\frametitle{Some unsolicited advice}
% 	\begin{itemize}
% 		\item	Recognize that the lectures are about 25\% of the work for this course
% 	\end{itemize}
% 	
% 
% 	Final year is a tough year: many competing demands on your time, especially in 2nd semester
% 
% 	Some useful tips:
% 	* exercise a few times per week: walk around campus, Pulse, group class, cardio circuit, jog, push-ups regularly
% 	* prioritize events in your social life and academic life
% 	* speak with your group members all the time about how feel: communicate communicate communicate
% 	* sleep
% 	* eat well: veggies and balance meals
% 	* overall time-management
% 		* yes, you will have to work on weekends, at least 1 day (no only on 4N, but others)
% 	\begin{exampleblock}{}
% 		Things not going well? Please communicate \emph{early}.
% 	\end{exampleblock}
% \end{frame}
% 
% \begin{frame}\frametitle{}
% 	CONTRIBUTE CONTINUOUSLY
% 		- Projects have many check points
% 		- Projects are too big to complete
% 	   	   in a couple of days
% 		- No Hitchhikers 
% 		- Peer Evaluations
% 		- Group-Instructor meetings
% 		
% 		- Discuss issues as they arise; don't wait until lots of pressure, then ``blow up''.  If needed, see instructor.
% 		
% 		- Could be cultural issues to overcome
% 		
% 		Group work is terrible.  I don't like my group members, some don't carry their share of the load, and some don't respect me.
% 		Symptoms:
% 			*	group norms are ignored
% 			*	poor attendance
% 			*	do not complete feedback forms completely
% 			*	unequal distribution of work
% 			*	group splits into ``camps''
% 			*	poor quality work, not really completed by deadline
% 			
% 			Grading on major projects will be based on group and individual contributions.  Peer evaluations count!
% 			
% 			Group must share the work
% \end{frame}
% 
% \begin{frame}\frametitle{Improper group collaboration}
% 	
% 	* Any between group sharing: electronic materials, documents
% 	* Between tutorial sessions
% 	* Only group members who contributed
% 	* 
% 	
% \end{frame}
% 
% \begin{frame}\frametitle{Over to you ...}
% 	Please tell me more about your hopes and expectations on the handout.
% \end{frame}