%!TEX root = ../4N4-course.tex

% \begin{frame}\frametitle{Plan for today's class}
% 	\begin{enumerate}
% 		\item	Thanks to TAs
% 		\item	Course evaluation reminder
% 		\item	Peer evaluation reminder
% 		\item	Course changes for 2013 based on feedback so far
% 		\item	Overview of the topics covered in 4N4
% 		\item	How to study for the final exam
% 	\end{enumerate}
% \end{frame}

\begin{frame}\frametitle{Special thanks to the TAs}
	\begin{itemize}
		\item	Myrto Korogiannaki
		\item	Heera Marway
		\item	Tyler Homer
	\end{itemize}
\end{frame}

\begin{frame}\frametitle{Administrative}
		Course evaluations: \href{https://evals.mcmaster.ca}{https://evals.mcmaster.ca}
		\begin{itemize}
			\item	We take these seriously
			\item	Suggestions for improvement are always welcome
			\item	Every year 4N4 has evolved significantly based on your ideas:
			\visible<2->{
				\vspace{12pt}
				\begin{itemize}
					\item	more weight for the midterm
					\item	peer evaluation
					\item	suggestion for a wireless mic for the videos
					\item	same SDL project for the entire class
					\item	tutorials used questions from the SDL project to build it up
					\item	switched the Safety and Operability topics around
					\item	more troubleshooting exposure: ``before'' and ``after'' tutorials
					\item	group meetings are more focused
					\item	good quality guest speakers
				\end{itemize}
			}
		\end{itemize}
\end{frame}

\begin{frame}\frametitle{Main topics covered in 4N4}
	\begin{center}
		\includegraphics[width=.75\textwidth]{\imagedir/econ-prob-solve/admin/4N4-logo-labelled-improved.png}
	\end{center}
	\vspace{-12pt}
	{\small {\color{brown}{This is a unique course: not taught anywhere else.}}}
\end{frame}

\begin{frame}\frametitle{What you've learned ...}
	\begin{exampleblock}{}
		\begin{center}
			\textbf{\large There's a whole lot more to a system}
		\end{center}
	\end{exampleblock}
	
	\begin{columns}[t]
		\column{0.55\textwidth}
	
			\begin{itemize}
				\item	Every unit you've learned about before has more complexity than you thought
				\item	Economics: how do costs scale?
				\item	Operability aspects of the unit
				\item	Safe operation of that unit?
				\item	How would you troubleshoot it?
			\end{itemize}
		\column{0.60\textwidth}
			\vspace{-20pt}
			\begin{center}
				\includegraphics[width=.75\textwidth]{\imagedir/econ-prob-solve/admin/4N4-logo-labelled-improved.png}
			\end{center}
	\end{columns}
	
	
	\vspace{12pt}
	And most importantly: {\color{myOrange}integrating these ideas across a whole flowsheet of units.}
\end{frame}

\begin{frame}\frametitle{What you've learned ...}
	\begin{exampleblock}{}
		\begin{center}
			\textbf{\large You are capable of more than you thought.}
		\end{center}
	\end{exampleblock}
	\begin{itemize}
		\item	Constant exposure to the material via group-based assignments helped you learn
		\item	Surprised at how previous courses were brought together
		\item	Improved your writing skills
		\item	Learned you can't do all the work yourself
		\item	Some realized that ``this'' isn't for them
	\end{itemize}
\end{frame}

\begin{frame}\frametitle{Group work}
	Some had tough dynamics: ``\emph{as strong as your weakest link}''
	
	\begin{center}
		\includegraphics[width=\textwidth]{\imagedir/econ-prob-solve/groups/working-in-groups-1.png}
	\end{center}
\end{frame}

\begin{frame}\frametitle{Group work}
	Most had successful collaboration
	\begin{center}
		\includegraphics[width=.65\textwidth]{\imagedir/econ-prob-solve/groups/working-in-groups-2.png}
	\end{center}


	Many enjoyed the rotating chairperson role.

\end{frame}

\begin{frame}\frametitle{Life-long learning / Self-directed learning}
	\begin{itemize}
		\item	Challenging; felt you were without guidance
		\item	Open-ended assignments and projects were a challenge
		\item	Forced to use group-work to complete them in time
		\item	You've become good at locating information required
		\item	Sorting out ``what's necessary'' from ``nice-to-have''
		\item	Become far more efficient at managing your time
	\end{itemize}
\end{frame}

\begin{frame}\frametitle{SDL after \texttt{4N04}}
	You will keep learning:
	\begin{itemize}
		\item	from the plant
		\item	running experiments
		\item	talking with experts
		\item	reading websites
		\item	company-sponsored courses, seminars and conferences 
		\item	reading books, journal publications, trade journals
	\end{itemize}
	
	\vspace{12pt}
	Remember: it was in the Code of Ethics. You are encouraged to do this, as well as encourage your colleagues and people working under you to do this.
\end{frame}

\begin{frame}\frametitle{Your course reflection}
	A course reflection will be posted
	\begin{itemize}
		\item	It is worth 5\% of your course grade
		\item	due by 17 December 2014, at 16:00
	\end{itemize}
	
	\vspace{24pt}
	{\color{myOrange}Questions asked:}
	\begin{itemize}
		\item	what 1 key piece of advice do you have for the 2015 students?
		\item	troubleshooting experience: round 1 compared to round 2
		\item	participation in class (active learning)
		\item	your skills learned with self-directed learning
		\item	group-work skills you learned
		\item	time-management skills you learned
		\item	surprising thing(s) learned in 4N
		\item	did you accomplish your goals for 4N4?
	\end{itemize}
\end{frame}

\begin{frame}\frametitle{What's in the exam}
	\begin{enumerate}
		\item	Engineering Economics
		\item	Operability
		\item	Process Safety
		\item	Troubleshooting
		\item	Engineering Professionalism and Ethics
		\item	Everything covered in tutorials, assignments and class
	\end{enumerate}
\end{frame}

\begin{frame}\frametitle{Economics: what we covered}
	\begin{enumerate}
		\item	Personal finance
		\item	Cash flows
		\item	Time value of money $\displaystyle F_n = \frac{C_n}{(1+i)^n}$
		\item	Profitability estimation: payback time; ROI; for independent projects we required DCFRR $\geq$ MARR and NPV $\geq$ 0
		\item	Tax and depreciation: \textbf{always taken into account}
		\item	Sensitivity analysis
		\item	Capital and operating cost estimation
		\item	{\color{red} Bring the list of CRA classes and CEPCI cost indices to the exam}
	\end{enumerate}
\end{frame}

\begin{frame}\frametitle{Operability}
	\begin{exampleblock}{}
		Recognize the plant must still operate under conditions, and in situations, different to what it was designed for.
	\end{exampleblock}
	\begin{enumerate}
		\item	\textbf{Operating window} at steady state
		\item	\textbf{Flexibility} and controllability: degrees of freedom; what's manipulated? what's controlled?
		\item	\textbf{Reliability}: parallel and series structures; duplicate units
		%\item	\textbf{Efficiency}: heat integration, recycle, different power sources
		\item	\textbf{Transitions}: maintenance, start-up and shut-down, regeneration, and grade changes. Bypass, batch-continuous interfaces; storage.
		\item	\textbf{Scheduling}: ideas related to batch sequencing and scheduling
	\end{enumerate}
\end{frame}

\begin{frame}\frametitle{Process Safety}
	\begin{columns}[t]
		\column{0.55\textwidth}
			\begin{enumerate}
				\item	Hierarchy
				\begin{enumerate}
					\item	BPCS
					\item	Alarms
					\item	SIS
					\item	Relief
					\item	Containment
					\item	Emergency response
				\end{enumerate}
				%\item	Preliminary analysis (checklists/relative ranking)
				\item	HAZOP: nodes, parameters and guide words
				\item	Case study: BP in Texas City
			\end{enumerate}
		\column{0.60\textwidth}
			\vspace{-20pt}
			\begin{center}
				\includegraphics[width=\textwidth]{\imagedir/econ-prob-solve/safety/safety-layers-Marlin.png}
			\end{center}
	\end{columns}
\end{frame}

\begin{frame}\frametitle{Troubleshooting}
	\begin{enumerate}
		\item	\textbf{Engage}: \emph{I want to and I can!}
		\item	\textbf{Define}: what is and isn't, fact \emph{vs} opinion; where do you want to be?
		\item	\textbf{Explore}: fundamentals, important variables, cause-and-effect
		\item	\textbf{Plan and diagnose}: root causes in a table, collect evidence, initiate diagnostic experiments (actions)
		\item	\textbf{Implement}: short-term and long-term solutions
		\item	\textbf{Look back}: reflection ... what worked and what didn't
	\end{enumerate}
	\vspace{12pt}
	You have seen 7 case studies for troubleshooting
	\begin{itemize}
		\item	More are available in Dr. Marlin's textbook
		\item	Two others appear under the ``Practice Problems'' section on the course website
	\end{itemize}
	Troubleshooting will appear in the exam.
\end{frame}

\begin{frame}\frametitle{Professionalism and Ethics}
	\begin{itemize}
		\item	The material we covered in class
		\item	The material posted on the course website
				\begin{itemize}
					\item	there are 11 extra case studies for you to practice with
					\item	detailed guidelines are given on how to approach the cases
				\end{itemize}
		\item	{\color{red} Bring the Code of Ethics sheet to the exam}
	\end{itemize}
\end{frame}

\begin{frame}\frametitle{Final exam}
	\begin{itemize}
		\item	13 December 2014 at 19:30 in MDCL/1105
		\item	Printed materials (textbooks, any papers, etc.) are allowed
		\item	Any calculator is allowed
		\item	Pencil is OK, as long as it is dark
		\item	Answer questions in any order
		\item	Answer each question on a new page
		\item	Please use bullet points to answer, where appropriate
		\item	Never repeat the question back in your answer
		\item	\textbf{Something is unclear, or seems incomplete, make a \emph{reasonable assumption} and continue with the question.}
	\end{itemize}
	\vspace{12pt}
	Preparing for the exam:
	\begin{itemize}
		\item	Please read Dr. Marlin's notes (not just slides, the notes)
		\begin{itemize}
			\item	Safety
			\item	Operability
			\item	Troubleshooting
		\end{itemize}
		\item	Please review the slides, videos, guest lectures, and material covered in class
	\end{itemize}
\end{frame}

\begin{frame}\frametitle{Thank you}
	\begin{itemize}
		\item	For your feedback in class, after class and anonymously
		\item	{\color{myOrange}{\textbf{Thank you for being a great class to teach. }}See you in 4C3 and 4G3.}
	\end{itemize}
\end{frame}

\begin{frame}\frametitle{}
	\begin{exampleblock}{}
		\begin{center}\huge {\color{purple}{\texttt{4W04} $-$ Bring it on\emph{!}  I'm ready\emph{!}}}
		\end{center}
	\end{exampleblock}
\end{frame}