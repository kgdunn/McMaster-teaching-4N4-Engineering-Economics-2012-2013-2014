%!TEX root = ../4N4-course.tex

\begin{frame}\frametitle{Plan for today's class}
	\begin{enumerate}
		\item	Thanks to TAs
		\item	Course evaluation reminder
		\item	Peer evaluation reminder
		\item	Course changes for 2013 based on feedback so far
		\item	Overview of the topics covered in 4N4
		\item	How to study for the final exam
	\end{enumerate}
\end{frame}

\begin{frame}\frametitle{Special thanks to the TAs}
	\begin{itemize}
		\item	Alicia Pascall
		\item	Yasser Ghobara
	\end{itemize}
\end{frame}

\begin{frame}\frametitle{Administrative}
	\begin{enumerate}
		\item	Course evaluations due: \href{https://evals.mcmaster.ca}{https://evals.mcmaster.ca} ... 55\% of you have filled it in: aiming for 70\%
		\item	Peer evaluation \& course reflection: \emph{see \href{https://docs.google.com/spreadsheet/viewform?formkey=dFE0M1J3NWstZjE5SE4tN19lRGxheGc6MQ}{course website for link}}
	\end{enumerate}
\end{frame}

\begin{frame}\frametitle{Main topics covered in 4N4}
	\begin{center}
		\includegraphics[width=.75\textwidth]{\imagedir/econ-prob-solve/admin/4N4-logo-labelled.png}
	\end{center}
	\vspace{-12pt}
	{\small {\color{brown}{This is a unique course: not taught anywhere else.}}}
\end{frame}

\begin{frame}\frametitle{What you've learned ...}
	\begin{exampleblock}{}
		\begin{center}
			\textbf{\large You are capable of more than you thought.}
		\end{center}
	\end{exampleblock}
	\begin{itemize}
		\item	Constant exposure to the material via assignments helped you learn
		\item	Surprised at how previous courses were brought together
		\item	Improved your presentation and writing skills
		\item	Can't do all the work yourself
		\item	Some realized that ``this'' isn't for them
		\item	Reputation of this course is a bit different from what you heard
	\end{itemize}
\end{frame}

\begin{frame}\frametitle{Group work}
	\begin{enumerate}
		\item	Some groups had tough topics
	\end{enumerate}
	\vfill
\end{frame}

\begin{frame}\frametitle{Group work}
	\begin{enumerate}
		\item	Some groups had tough topics
		\item	Some had tough dynamics: ``\emph{as strong as your weakest link}''
	\end{enumerate}
	\begin{center}
		\includegraphics[width=\textwidth]{\imagedir/econ-prob-solve/groups/working-in-groups-1.png}
	\end{center}
\end{frame}

\begin{frame}\frametitle{Group work}
	\begin{enumerate}
		\item	Some groups had tough topics
		\item	Most had successful collaboration
	\end{enumerate}
	\begin{center}
		\includegraphics[width=.65\textwidth]{\imagedir/econ-prob-solve/groups/working-in-groups-2.png}
	\end{center}

	\begin{enumerate}
		\setcounter{enumi}{2}
		\item	Many enjoyed the rotating chairperson role.
	\end{enumerate}
\end{frame}

\begin{frame}\frametitle{Life-long learning / Self-directed learning}
	\begin{itemize}
		\item	Challenging; felt you were without guidance
		\item	Open-ended assignments and projects were a challenge
		\item	Forced to use group-work to complete them in time
		\item	You've become good at locating information required
		\item	Sorting out ``what's necessary'' from ``nice-to-have''
		\item	Become more efficient at managing your time
	\end{itemize}
\end{frame}

\begin{frame}\frametitle{What's in the exam}
	\begin{enumerate}
		\item	Small section on engineering economics
		\item	Process Safety
		\item	Operability
		\item	Troubleshooting
		\item	Everything else covered in tutorials, assignments and class
	\end{enumerate}
\end{frame}

\begin{frame}\frametitle{Economics: what we covered}
	\begin{enumerate}
		\item	Personal finance
		\item	Cash flows
		\item	Time value of money $\displaystyle F_n = \frac{C_n}{(1+i)^n}$
		\item	Profitability estimation: payback time; for independent projects we required DCFRR $\geq$ MARR and NPV $\geq$ 0
		\item	Tax and depreciation: \textbf{always taken into account}
		\item	Sensitivity analysis
		\item	Capital and operating cost estimation
	\end{enumerate}
\end{frame}

\begin{frame}\frametitle{Process Safety}
	\begin{columns}[t]
		\column{0.55\textwidth}
			\begin{enumerate}
				\item	Hierarchy
				\begin{enumerate}
					\item	BPCS
					\item	Alarms
					\item	SIS
					\item	Relief
					\item	Containment
					\item	Emergency response
				\end{enumerate}
				\item	Preliminary analysis (checklists/relative ranking)
				\item	HAZOP: nodes, parameters and guidewords
				\item	Case study: BP, Texas City
			\end{enumerate}
		\column{0.60\textwidth}
			\vspace{-20pt}
			\begin{center}
				\includegraphics[width=\textwidth]{\imagedir/econ-prob-solve/safety/safety-layers-Marlin.png}
			\end{center}
	\end{columns}	
\end{frame}

\begin{frame}\frametitle{Operability}
	\begin{exampleblock}{}
		Recognize the plant must still operate under conditions, and in situations, different to what it was designed for.
	\end{exampleblock}
	\begin{enumerate}
		\item	\textbf{Operating window} at steady state
		\item	\textbf{Flexibility} and controllability: degrees of freedom; what's manipulated? what's controlled?
		\item	\textbf{Reliability}: parallel and series structures; duplicate units
		\item	\textbf{Efficiency}: heat integration, recycle, different power sources
		\item	\textbf{Transitions}: maintenance, start-up and shut-down, regeneration, and grade changes. Bypass, batch-continuous interfaces; storage.
	\end{enumerate}
\end{frame}

\begin{frame}\frametitle{Troubleshooting}
	\begin{enumerate}
		\item	\emph{I want to and I can!}
		\item	\textbf{Define}: what is and isn't, where, when, who, what
		\item	\textbf{Explore}: fundamentals, important variables, cause-and-effect
		\item	\textbf{Plan and diagnose}: root causes in a table, collect evidence, initiate diagnostic experiments (actions)
		\item	\textbf{Implement}: short-term and long-term solutions
		\item	\textbf{Look back}: reflection ... what worked and what didn't
	\end{enumerate}
\end{frame}

\begin{frame}\frametitle{Final exam}
	\begin{itemize}
		\item	17 December 2012 at 12:00
		\item	Printed materials (textbooks, any papers, etc.) are allowed
		\item	Any calculator is allowed
		\item	Pencil is OK, as long as it is dark
		\item	Answer questions in any order
		\item	Please use bullet points to answer
		\item	Never repeat the question back in your answer
		\item	Something is unclear, or seems incomplete, make a \emph{reasonable assumption} and continue with the question.
	\end{itemize}
	\vspace{12pt}
	Preparing for the exam:
	\begin{itemize}
		\item	Please read Dr. Marlin's notes (not slide, the notes)
		\item	Please review the slides, videos and material covered in class
	\end{itemize}
\end{frame}

\begin{frame}\frametitle{Thank you}
	\begin{itemize}
		\item	For your patience with delays in returning graded material
		\item	For your feedback via the mid-term evaluations and anonymous evaluations
		\item	Some changes planned in 2013:
		\begin{itemize}
			\item	More frequent required group ``check-ins'' with me
			\item	More time on troubleshooting and operability
			\item	Adjust presentation schedule and timing
			\item	Less emphasis on distillation columns, oil and gas systems; more bio, foods, pharma, pulp and paper, polymers
			\item	Integrate the Seider \emph{et al.} text book in 2013
			\item	Course will become more integrated with 4W4
			\item	Probably about the same amount of work and assignments in 2013 as this year
		\end{itemize}
		
		\vspace{48pt}
		\pause
		\item	{\color{myOrange}{\textbf{Thank you for being a great class to teach. }}See you in 4C3.}
	\end{itemize}
\end{frame}

\begin{frame}\frametitle{}
	\begin{exampleblock}{}
		\begin{center}\huge {\color{purple}{\texttt{4W04} $-$ Bring it on\emph{!}  I'm ready\emph{!}}}
		\end{center}
	\end{exampleblock}
\end{frame}