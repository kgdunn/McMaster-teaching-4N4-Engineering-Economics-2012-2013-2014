% Compile once, then ``bibtex 2013-4N4-class-12``, then compile twice more
\begin{frame}\frametitle{Credit} These slides are all the work of Chris Ewaschuk
	
	\vspace{12pt} TA for the 4N04 course in 2013. 
\end{frame}

\begin{frame}\frametitle{Practice Example}
	Consider the simplified Kondili Process and create a feasible production schedule for a 4 hour time horizon. A complete production sequence requires the following steps: 
	\begin{enumerate}
		\item	Heat A (1hr), 
		\item	React A, B and C (2hr), 
		\item	Product separation (1h) 
	\end{enumerate}
	\begin{figure}
		[!htb] 
		\begin{center}
			\includegraphics[width=0.75
			\textwidth]{\imagedir/econ-prob-solve/batch-sequencing/SKPFD.png}
			
			\label{fig:SKPFD} 
		\end{center}
	\end{figure}
\end{frame}

\begin{frame}\frametitle{Practice Example II}
	Considerations: 
	\begin{itemize}
		\item	Do you have a feasible production schedule? 
		\item	What happens if one of the units breaks down or requires maintenance? 
		\item	Have you considered unit capacities? Storage policies? 
		\item	is your schedule ``good''? {\scriptsize \emph{We will define good later. }}
		\item	How far to the limit can we push the schedule? 
	\end{itemize}
	\begin{figure}
		[!htb] 
		\begin{center}
			\includegraphics[width=0.65\textwidth]{\imagedir/econ-prob-solve/batch-sequencing/SKPFD.png}
			\label{fig:SKPFD2} 
		\end{center}
	\end{figure}
\end{frame}

\begin{frame}\frametitle{What is Batch Scheduling?}
	\begin{itemize}
		\item	batch scheduling can be considered the sequencing and timing of process unit utilization in order to achieve a feasible production schedule		
		\item	{\color{purple}{\textbf{scheduling}}} - timing of events 
		\item	{\color{purple}{\textbf{sequencing}}} - the order events are placed in (based on precedence relationships)		
		\item	answers the questions of when and how a unit operation will be used in order to fulfill a company's needs
		\item	operational plant hierarchy:
	\end{itemize}
	\begin{figure}
		[!htb] 
		\begin{center}
			\includegraphics[width=0.6\textwidth]{\imagedir/econ-prob-solve/batch-sequencing/PlantHierarchy.png} \label{fig:PH} 
		\end{center}
	\end{figure}
\end{frame}

\begin{frame}\frametitle{What is Batch Scheduling? II}
	\begin{itemize}
		\item	It is a critical issue in process operations 
		\item	Short-term scheduling in batch processes involves:
		\begin{itemize}
			\item	allocation decisions 
			\item	limited resources 
			\item	a given time horizon 
			\item	manufacturing products through the use of a batch recipe \cite{MendyNco06} 
		\end{itemize}
	\end{itemize}
\end{frame}

\begin{frame}\frametitle{Problem Classification - Objectives}
	
	Various objectives / metrics exist:
	\begin{itemize}
		
		\item	Maximize Profit		
		\item	Minimize Cost		
		\item	Minimize {\color{purple}Makespan} - given a fixed demand, minimize total required production time 
		\newline
		
		\item	Maximize Throughput - given a fixed time horizon, produce as much as possible 
		\newline
		
		\item	{\color{purple}Tardiness} - absolute value of the difference between completion time and due date 
		\newline
		
		\item	{\color{purple}Lateness} - difference between completion time and due date
	\end{itemize}
\end{frame}

\begin{frame}\frametitle{Why is this a relevant problem to Chemical Engineers?}
	\begin{itemize}
		
		\item	A typical medium scale integrated chemical site produce revenues worth \$ 5 to 10 million per day \cite{ShobW} 
		\item	Exxon Chemicals reduced annual operating costs by 2 \% and operating inventory by 20 \% \cite{ShobW} 
		\item	DuPont reduced working capital in inventory from \$ 160m to 95m \cite{ShobW} 
		\item	Scheduling is being used more and more commonly as a competitive advantage to cut costs. The increasingly relevant question is: ``{\color{myOrange}\emph{How can I out-schedule my competitors}}''?
	\end{itemize}
\end{frame}

\begin{frame}\frametitle{Why is this a relevant problem to Chemical Engineers? II}
	Certain processes are still operated batch-wise for several reasons i.e.:
	\begin{itemize}
		\item	retrofitting existing batch tech or commissioning new continuous processes can have a very high capital cost			
		\item	industries that are highly regulated (Pharma) may not be able to currently transition from batch to continuous due to governmental (FDA) restrictions - currently a growing topic in ChE research			
		\item	quality control issues - if one batch is ruined or sub-standard you may be able to prevent a fault from affecting other batches (avoid spoilage)			
		\item	your product yields sufficiently high profit margins with batch tech that converting to continuous is not necessarily desirable			
		\item	batch processes can be more flexible than continuous i.e. batch can allow for multiple process configurations whereas continuous might not
	\end{itemize}
\end{frame}

\begin{frame}\frametitle{Scheduling Methods - Heuristical}
	\begin{itemize}
		\item	Many companies may perform process scheduling using simply a program like Excel, experience, and intuition much like you might have in the previous example		
		\item	Let's look at some other methods		
		\item	Heuristic methods take advantage of empirical solution methods \cite{MendyNco06}
	\end{itemize}
	\begin{itemize}
		\item	{\color{purple}{\textbf{cycle time}}} is the average time to produce a batch 
	\end{itemize}
\end{frame}

\begin{frame}\frametitle{Scheduling Example I}
	
	Non-overlapping schedules \; Backward Overlapping Schedule
	\begin{figure}
		[!htb] 
		\begin{center}
			\includegraphics[width=0.45
			\textwidth]{\imagedir/econ-prob-solve/batch-sequencing/non-overlapping-sequence}
			
			%\caption{Optimal Schedule}
			
			\includegraphics[width=0.45
			\textwidth]{\imagedir/econ-prob-solve/batch-sequencing/backward-shift-to-get-overlap}
			
			%\caption{Optimal Schedule}
			\label{fig:Eg1} 
		\end{center}
	\end{figure}
	
	Overlapping schedules
	\begin{figure}
		[!htb] 
		\begin{center}
			\includegraphics[width=0.75
			\textwidth]{\imagedir/econ-prob-solve/batch-sequencing/limiting-case-for-overlap}
			
			%\caption{Optimal Schedule}
			\label{fig:Eg2} 
		\end{center}
	\end{figure}
\end{frame}

\begin{frame}\frametitle{Scheduling Example II}
	\begin{figure}
		[!htb] 
		\begin{center}
			\includegraphics[width=0.99
			\textwidth]{\imagedir/econ-prob-solve/batch-sequencing/example-3-5-table}
			
			%\caption{Optimal Schedule}
			\label{fig:Eg3} 
		\end{center}
	\end{figure}
	\begin{figure}
		[!htb] 
		\begin{center}
			\includegraphics[width=0.99
			\textwidth]{\imagedir/econ-prob-solve/batch-sequencing/flowshop-plant-for-ABC}
			
			%\caption{Optimal Schedule}
			\label{fig:Eg3pfd} 
		\end{center}
	\end{figure}
\end{frame}

\begin{frame}\frametitle{Scheduling Example A}
	\begin{figure}
		[!htb] 
		\begin{center}
			\includegraphics[width=0.99
			\textwidth]{\imagedir/econ-prob-solve/batch-sequencing/example-3-5-A-only}
			
			%\caption{Optimal Schedule}
			\label{fig:Eg4} 
		\end{center}
	\end{figure}
\end{frame}

\begin{frame}\frametitle{Scheduling Example B}
	\begin{figure}
		[!htb] 
		\begin{center}
			\includegraphics[width=0.99
			\textwidth]{\imagedir/econ-prob-solve/batch-sequencing/example-3-5-B-only}
			
			%\caption{Optimal Schedule}
			\label{fig:Eg5} 
		\end{center}
	\end{figure}
\end{frame}

\begin{frame}\frametitle{Scheduling Example C}
	\begin{figure}
		[!htb] 
		\begin{center}
			\includegraphics[width=0.99
			\textwidth]{\imagedir/econ-prob-solve/batch-sequencing/example-3-5-C-only}
			
			%\caption{Optimal Schedule}
			\label{fig:Eg5} 
		\end{center}
	\end{figure}
\end{frame}

\begin{frame}\frametitle{Scheduling Example A,B,C}
	\begin{figure}
		[!htb] 
		\begin{center}
			\includegraphics[width=0.99
			\textwidth]{\imagedir/econ-prob-solve/batch-sequencing/example-3-5-A-B-C}
			
			%\caption{Optimal Schedule}
			\label{fig:Eg5} 
		\end{center}
	\end{figure}
\end{frame}

\begin{frame}\frametitle{Problem classification: Plant type}
	
	{\color{purple}Flowshop}
	\begin{itemize}
		\item	all products use same equipment or same sequence 
		\item	i.e. the steps to make biscuits or cake include (mix, bake, cool, icing) 
	\end{itemize}
	\begin{figure}
		[!htb] 
		\begin{center}
			\includegraphics[width=0.99
			\textwidth]{\imagedir/econ-prob-solve/batch-sequencing/flowshop-plant-for-ABC}
			
			%\caption{Optimal Schedule}
			\label{fig:flow} 
		\end{center}
	\end{figure}
\end{frame}

\begin{frame}\frametitle{Problem classification: Plant type}
	
	{\color{purple}Jobshop}
	\begin{itemize}
		\item	products use either different equipment or sequence\\
	\end{itemize}
	\begin{figure}
		[!htb] 
		\begin{center}
			\includegraphics[width=0.99
			\textwidth]{\imagedir/econ-prob-solve/batch-sequencing/two-examples-of-jobshop-plants}
			
			%\caption{Optimal Schedule}
			\label{fig:job} 
		\end{center}
	\end{figure}
	
	\cite{turton}
\end{frame}

\begin{frame}\frametitle{Scheduling methods: Optimization related}
	
	Constraint Programming 
	\begin{itemize}
		\item	was originally developed to solve feasibility problems 
		\item	has been extended to solve optimization problems 
		\item	contains continuous, integer, and boolean variables 
		\item	variables can be indexed by other variables \cite{MendyNco06} 
	\end{itemize}
	
	Metaheuristics
	\begin{itemize}
		\item	local search algorithms such as Simulated Annealing, Genetic Algorithms, Tabu Search \cite{MendyNco06} 
		\item	can obtain good quality solutions within reasonable time 
		\item	do not guarantee optimality 
		\item	give no measure of the optimality gap 
	\end{itemize}
\end{frame}

\begin{frame}\frametitle{Scheduling Methods - Optimization Based} Use an optimization framework i.e. the following Mixed Integer Linear Program (MILP)
	\begin{equation*}
		\begin{aligned}
			& \underset{X,Y}{\text{minimize}} & & C_{1}^{T}X + C_{2}^{T}Y \\
			& \text{subject to} & & A^{T}X + B^{T}Y \leq D\\
			& Y \; = \; Integer 
		\end{aligned}
	\end{equation*}
	\begin{itemize}
		\item	can be solved to global optimality 
		\item	give a measure of optimality gap 
		\item	can be very computationally expensive to solve 
		\item	modelling can be very complicated 
	\end{itemize}
\end{frame}

\begin{frame}\frametitle{Scheduling Methods - Optimally Generated Schedules}
	\begin{figure}
		[!htb] 
		\begin{center}
			\includegraphics[width=0.99
			\textwidth]{\imagedir/econ-prob-solve/batch-sequencing/OG1} \label{fig:Og1} 
		\end{center}
	\end{figure}
	\begin{center}
		\begin{tabular}
			{ c |c|c| c } \hline Binary Variables & 78 & Continuous Variables & 167\\
			\hline Constraints & 247 & Nodes & 0\\
			\hline MIP Simplex Iterations & 81 & CPU seconds & 0.031\\
			\hline 
		\end{tabular}
	\end{center}
\end{frame}

\begin{frame}\frametitle{Scheduling Methods - Optimally Generated Schedules II}
	
	The optimizations presented here are quite ``small'' problems and solve very fast\\
	
	\cite{FnL04} cite some models with thousands of binaries, continuous variables, and constraints which can take between 20 minutes up to 3 hours!
	\begin{figure}
		[!htb] 
		\begin{center}
			\includegraphics[width=0.99
			\textwidth]{\imagedir/econ-prob-solve/batch-sequencing/OG2}  \label{fig:Og2} 
		\end{center}
	\end{figure}
\end{frame}

\begin{frame}\frametitle{Scheduling Methods - Optimally Generated Schedules III}
	
	\cite{ierpfa} \nocite{ierpfb}
	\begin{figure}
		[!htb] 
		\begin{center}
			\includegraphics[width=0.99
			\textwidth]{\imagedir/econ-prob-solve/batch-sequencing/OG3} \label{fig:Og3} 
		\end{center}
	\end{figure}
	\begin{center}
		\begin{tabular}
			{ c |c|c| c } \hline Binary Variables & 60 & Continuous Variables & 192\\
			\hline Constraints & 346 & Nodes & 0\\
			\hline MIP Simplex Iterations & 137 & CPU seconds & 0.187\\
			\hline 
		\end{tabular}
	\end{center}
\end{frame}

\begin{frame}\frametitle{Scheduling Software}
	\begin{itemize}
		\item	in terms of optimization based scheduling CPLEX (ILOG), XPRESS (Dash Optimization, 2003), and Gurobi are available which use LP-based branch and bound algorithm combined with cutting plane techniques in combination with GAMS, AMPL, AIMMS (ChE 4G03) 
		\item	Aspen Plant Scheduler where the solution is integrated with the Aspen Available-to-Promise/Capable-to-Promise solution, 
		\item	OSS scheduler from Process Systems Enterprise Ltd. determines optimal production based on STN as a basis 
		\item	VirtECS Schedule from Advanced Process Combinatorics devises an optimized schedule satisfying constraints and includes an Interactive Scheduling Tool 
		\item	SAP Advanced Planner and Optimizer SAP uses mySAP (SAP APO), which supports real-time and network optimization 
	\end{itemize}
\end{frame}

\begin{frame}\frametitle{Semi Exhaustive List of Batch Scheduling Applications - Research Collaborations with Carnegie Mellon}
	\begin{itemize}
		\item	Batch Scheduling with Electric Power Constraints (ABB) 
		\item	Multiperiod Scheduling of Polypropylene Production (Braskem) 
		\item	Simultaneous Scheduling and Dynamic Optimization of Batch Processes (Dow) 
		\item	Global Optimization of Bilinear GDP Models (ExxonMobil) 
		\item	Multistage Stochastic Programming for Design and Planning of Oil and Gasfields (ExxonMobil) 
		\item	Planning and Scheduling for Glass Production (PPG) 
		\item	Capacity Planning of Power Intensive Networks with Changing Electricity Prices (Praxair) 
		\item	Scheduling of Crude Oil Operations (Total) 
		\item	Scheduling of Fast Moving Consumer Goods (Unilever) 
	\end{itemize}
	\cite{Gman-adv} 
\end{frame}

\begin{frame}\frametitle{Semi Exhaustive List of Batch Scheduling Applications - II Industry Breakdown}
	
	Petroleum 
	\begin{itemize}
		\item	BP, Braskem, Ecopetrol, ExxonMobil, NOVA Chemicals, Total 
	\end{itemize}
	
	Vendors/ Consulting 
	\begin{itemize}
		\item	ABB, Honeywell 
	\end{itemize}
	
	Polymer/ Chemicals Manufacturing 
	\begin{itemize}
		\item	DOW, DuPont, Braskem 
	\end{itemize}
	
	Air Separation 
	\begin{itemize}
		\item	Praxair, Air Liquide 
	\end{itemize}
	
	Consumer Goods 
	\begin{itemize}
		\item	Unilever 
	\end{itemize}
	
	Glass Manufacturing 
	\begin{itemize}
		\item	PPG 
	\end{itemize}
	
	and more! 
\end{frame}

\begin{frame}\frametitle{Short History Lesson}
	\begin{itemize}
		\item	scheduling was a problem originally considered and studied by Operations Research and Management Science (Fields of Business - Logistics) 
		\item	Gantt Chart invented by Henry Gantt in (Organizing for Work, 1910) 
		\item	Gantt chart is a bar plot illustrating the start and finish of a project schedule 
	\end{itemize}
	\begin{figure}
		[!htb] 
		\begin{center}
			\includegraphics[width=0.9
			\textwidth]{\imagedir/econ-prob-solve/batch-sequencing/Gchart}
			
			%\caption{Gantt Chart}
			\label{fig:Gchart} 
		\end{center}
	\end{figure}
\end{frame}

\begin{frame}\frametitle{Short History Lesson II}
	
	A project management technique referred to as the Critical Path Method (CPM) that uses the Project Evaluation and Review Technique (PERT) from 1950's
	\begin{itemize}
		\item	maps out the activities required to complete a project 
		\item	includes the time it will take to complete activities and activity interdependence 
		\item	circles (nodes) represent an activity requiring completion, arcs represent dependencies 
	\end{itemize}
\end{frame}

\begin{frame}\frametitle{Short History Lesson III}
	\begin{itemize}
		\item	the critical path (shortest path that completes all activities) is the longest path through the schematic (minimax problem) 
		\item	below is the activity on node (AON) configuration, activity on arc is an alternative (AOA) 
		\item	a PERT chart:
	\end{itemize}
	\begin{figure}
		[!htb] 
		\begin{center}
			\includegraphics[width=0.9
			\textwidth]{\imagedir/econ-prob-solve/batch-sequencing/PERT} \label{fig:PERT} 
		\end{center}
	\end{figure}
\end{frame}

\begin{frame}\frametitle{Short History Lesson IV}
	\begin{itemize}
		\item	scheduling was an increasingly relevant industrial problem and was adopted by industrial and chemical engineers 
		\item	one of the first academic papers in Chemical Engineering was by \cite{kondili} and \cite{shah} who developed the concept of the State Task Network (STN) 
		\item	the STN is a systematic method by which Chemical Engineers can model a chemical process in an optimization framework (This is and was a SIGNIFICANT contribution!) 
	\end{itemize}
	\begin{center}
		ChE vs. OR terminology
		\begin{tabular}
			{ l | c | r } \hline & ChE & OR \\
			\hline i & tasks & jobs \\
			j & units & machines \\
			n & time intervals / event points & time intervals \\
			\hline 
		\end{tabular}
	\end{center}
\end{frame}

\begin{frame}\frametitle{Short History Lesson V}
	\begin{itemize}
		\item	state nodes represent feeds, intermediate and final products (circles) 
		\item	task nodes, represent processing operations which transform material from one or more input states to one or more output states (rectangles) 
	\end{itemize}
	\begin{figure}
		[!htb] 
		\begin{center}
			\includegraphics[width=0.9
			\textwidth]{\imagedir/econ-prob-solve/batch-sequencing/STN} \label{fig:STN} 
		\end{center}
	\end{figure}
\end{frame}

\begin{frame}\frametitle{Short History Lesson VI}
	\begin{itemize}
		\item	\cite{EWO} describes Enterprise-wide optimization an area that lies at the interface of chemical engineering (process systems engineering)and operations research 
		\item	involves optimizing the operations manufacturing (batch or continuous) and distribution 
		\item	predicts major involvement of chemical engineers to develop novel novel computational models and algorithms to solve real world problems \cite{EWO} 
	\end{itemize}
\end{frame}

{\bf References}
\bibliography{summary} 
\bibliographystyle{apalnew} 
