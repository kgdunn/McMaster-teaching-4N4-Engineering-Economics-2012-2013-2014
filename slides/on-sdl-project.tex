\begin{frame}\frametitle{What we will cover}
	\begin{center}
		\includegraphics[width=\textwidth]{\imagedir/econ-prob-solve/outcomes/econ-prob-solve-4N4-mindmap.png}
	\end{center}
\end{frame}

\begin{frame}\frametitle{Self-directed learning (SDL)}
	Any examples of self-directed learning you've succeeded at recently?
	\begin{itemize}
		\item	learn how to ride a motorcycle
		\item	\pause\iftoggle{instructor}{how to fix drywall}{}
		\item	\iftoggle{instructor}{change your car's oil or battery}{}		
		\item	\iftoggle{instructor}{fix plumbing in your apartment/house}{}
		\item	\iftoggle{instructor}{cook an ethnic food dish to impress your date/partner/family}{}
		\item	\iftoggle{instructor}{something learned during your co-op work term?}{}
		\item	\iftoggle{instructor}{going through process of buying a car or a house}{}
		\item	\iftoggle{instructor}{plant, grow and maintain your own vegetables}{}
		\item	\iftoggle{instructor}{learn a new language for travel/pleasure}{}
		\item	\iftoggle{instructor}{start your own company and run it: what is required?}{}
		\item	\iftoggle{instructor}{figure out if I'm better off buying a new car or a used car?}{}
	\end{itemize}
	\vspace{12pt}
	\pause
	Our 1st tutorial, next Monday, on personal finance: {\small \color{myOrange}{completely SDL}}
\end{frame}

\begin{frame}\frametitle{Why self-directed learning is important}
	\begin{itemize}
		\item	No employer has an obligation to employ you until retirement
		\item	You must take action to get employment:
		\begin{itemize}
			\item	co-op terms
			\item	job market is looking for new hires and you have the profile they are looking for
			\item	you could also start your own company
		\end{itemize}
		\pause
		\item	You will likely be a contract-based employee for the next few years (e.g. I'm still a contracted employee at McMaster)
		\item	Your degree is no guarantee for employment: but it helps tremendously: {\color{myRed}{{\small you don't need to remain in chemical engineering!}}}
		\pause
		\item	Once you start working: \textbf{keep yourself employable}
		\begin{itemize}
			\item	technologies currently used that weren't taught in engineering
			\item	new technologies will be invented after you graduate
			\item	new/different ways of doing business			
		\end{itemize}
	\end{itemize}
\end{frame}

\begin{frame}\frametitle{SDL project}
	Apply what you have learned and are currently learning to a process.
	\vspace{12pt}
	\begin{itemize}
		\item	economics
		\item	operability issues: startup, operation, and shut down
		\item	safety
		\item	troubleshooting
		\item	summarize your learning (reflection on what was learned)
	\end{itemize}
\end{frame}

\begin{frame}\frametitle{SDL in this course}
	Self-directed learning is a process: 
	\begin{itemize}
		\item	identify what you already know
		\item	recognize where you don't have the knowledge
		\item	diagnose what you need
		\item	figure out how to meet your needs (which resources to use?)
		\item	implement it: 
		\begin{itemize}
			\item	go to the library
			\item	multiple internet searches
			\item	meet with people
			\item	attend conferences
			\item	phone up colleagues for informational interviews
		\end{itemize}
		\item	evaluate the success: did it work? do I need to start over again?
	\end{itemize}
\end{frame}

\begin{frame}\frametitle{What SDL is not}
	\begin{itemize}
		\item	Not a chance for me to slack off and you do the work!
		\item	Not where you find information to support a single hypothesis {\color{myOrange}{{\small (you should seek out multiple opinions)}}}
		\item	Not going to find your answers the first time
		\item	Not always going to succeed in achieving your goal, but you will learn along the way
		\item	Your group members are not competitors
	\end{itemize}
\end{frame}

\begin{frame}\frametitle{The role of the TAs and myself}
	\begin{itemize}
		\item	Give guidance at review meetings
		\item	We provide a respectful and positive manager-colleague relationship
		\item	We can help as a resource and for referrals
		\item	Help evaluate your learning	
	\end{itemize}	
\end{frame}

\begin{frame}\frametitle{Why group-based work}
	\begin{itemize}
		\item	Groups of 5 members will be created
		\item	$1+1+1+1+1 > 5$: magnify your strengths
		\item	You will always work in groups after graduation
		\item	Why do companies require group (team)work? \seefull{Turton, Ch 28}
		\begin{itemize}
			\item	Variety of viewpoints and expertise are brought together
			\item	Time constraints
		\end{itemize}	
		\vspace{12pt}
		\pause
		Our case might be a little unrealistic though:
		\begin{itemize}
			\item	You might often be the only (chemical) engineer on a team in a company
			\item	But that means you have to know \textbf{all your stuff}
		\end{itemize}
	\end{itemize}
\end{frame}

\begin{frame}\frametitle{Working in groups: less of this}
	\begin{center}
		\includegraphics[width=\textwidth]{\imagedir/econ-prob-solve/groups/working-in-groups-1.png}
	\end{center}
\end{frame}

\begin{frame}\frametitle{Signs of bad group dynamics}
	\begin{exampleblock}{}
		``Group work is terrible. I don't like my group members, some don't carry their share of the load, and some don't respect me.''
	\end{exampleblock}
	Symptoms:
	\begin{itemize}
		\item	group norms are ignored
		\item	poor attendance
		\item	unequal distribution of work
		\item	group splits into ``camps''
		\item	poor quality work, not really completed by deadline
	\end{itemize}
\end{frame}

\begin{frame}\frametitle{Groups: we hope you only experience this}
	\begin{center}
		\includegraphics[width=0.80\textwidth]{\imagedir/econ-prob-solve/groups/working-in-groups-2.png}
		\\
		$\text{Group} \longrightarrow \text{Team}$
	\end{center}	
\end{frame}

\begin{frame}\frametitle{The role of your group members}
	\begin{itemize}
		\item	They support your learning process: help identify goals
		\item	Accommodate each other's schedules		
		\item	Are resources themselves
		\item	Are not competitors, but become a team
	\end{itemize}
\end{frame}

\begin{frame}\frametitle{Continuous group contribution}
	\begin{itemize}
		\item	Projects have many check points
		\item	Projects are too big to complete in a couple of days
		\item	No hitchhikers
		\item	Peer evaluations will be used and applied
		\item	Group-instructor or group-TA meetings to gauge contribution
		\item	Discuss issues as they arise; don't wait until lots of pressure, then ``blow up''.  If needed, see instructor.
		\item	There might be cultural issues to overcome.
	\end{itemize}
\end{frame}

\begin{frame}\frametitle{Improper group collaboration}
	\begin{itemize}
		\item	Any between-group sharing: electronic materials, documents
		\item	including sharing materials between tutorial groups A and B
		\item	Adding group members names that did not contribute
	\end{itemize}
\end{frame}

\begin{frame}\frametitle{Group tips}
	\begin{itemize}
		\item	Make full use of collaboration tools such as
		\begin{itemize}
			\item	Google Docs
			\item	Zoho
			\item	Microsoft SkyDrive
			\item	Skype
			\item	Messenger
		\end{itemize}
	\end{itemize}
\end{frame}

\begin{frame}\frametitle{Group norms}
	How will your group deal with:
	\begin{itemize}
		\item	Absent member with excuse
		\item	Absent member without excuse
		\item	Personality conflicts
		\item	Not 100\% agreement
		\item	No one wants to take the chairperson role (fear?)
		\item	Person(s) with low motivation to work
		\item	Uneven work distribution
		\item	Lack of respect: blaming/insulting a person
		\item	Accepting constructive and helpful criticism
		\item	Accept group responsibility for failure
		\item	Avoiding ``groupthink''
	\end{itemize}
\end{frame}
